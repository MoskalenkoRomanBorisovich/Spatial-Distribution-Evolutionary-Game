\documentclass[a4paper,12pt]{article}

%%% Работа с русским языком
\usepackage{cmap}					% поиск в PDF
\usepackage{mathtext} 				% русские буквы в фомулах
\usepackage[T2A]{fontenc}			% кодировка
\usepackage[utf8]{inputenc}			% кодировка исходного текста
\usepackage[english,russian]{babel}	% локализация и переносы

%%% Дополнительная работа с математикой
\usepackage{amsfonts,amssymb,amsthm,mathtools} % AMS
\usepackage{amsmath}
\usepackage{icomma} % "Умная" запятая: $0,2$ --- число, $0, 2$ --- перечисление

%% Номера формул
%\mathtoolsset{showonlyrefs=true} % Показывать номера только у тех формул, на которые есть \eqref{} в тексте.

\usepackage{hyperref}
\hypersetup{
    colorlinks=true,
    linkcolor=blue,
    filecolor=magenta,      
    urlcolor=cyan,
}

\usepackage{float}


%% Шрифты
\usepackage{euscript}	 % Шрифт Евклид
\usepackage{mathrsfs} % Красивый матшрифт

%%% Работа с картинками
\usepackage{graphicx}  % Для вставки рисунков
\graphicspath{{images/}{images2/}}  % папки с картинками
\setlength\fboxsep{3pt} % Отступ рамки \fbox{} от рисунка
\setlength\fboxrule{1pt} % Толщина линий рамки \fbox{}
\usepackage{wrapfig} % Обтекание рисунков и таблиц текстом
\usepackage{caption}
\captionsetup{labelsep=period} %. вместо : в рис

\title{Ускорение программ на Python с помощью Cython и numba}
\author{Москаленко Роман Борисович}
\date{May 2020}

\begin{document}

\maketitle

\section{Введение}

В проекте по моделированию пространственно-распределённых игр основной задачей являлось написание кода для моделирования игры. Мы моделируем игру Новака-Мея, основанную на дилемме заключённого. В оригинальном варианте игра происходит на плоском квадратном поле, в каждой клетке поля расположены агенты. Все  агенты следуют одной из двух стратегий: $\mathbf{C}$ - агент сотрудничает со всеми, $\mathbf{D}$ - агент отказывается сотрудничать. На каждом шаге игры все агенты играют со своими соседями и с собой (всего 9 игр для каждого человека), агенты, находящиеся у края поля играют с агентами с другой стороны поля. Агенты получают очки согласно таблице.
\[
    \begin{tabular}{|c|c|c|c|}
    \hline
    \multicolumn{2}{|c|}{стратегия} & \multicolumn{2}{c|}{счёт}\\
    агент №1 & агент №2 & агент №1 & агент №2 \\
    \hline
    $\mathbf{C}$ & $\mathbf{C}$ & 1 & 1 \\
    \hline
    $\mathbf{C}$ & $\mathbf{D}$ & 0 & $\mathbf{b}$ \\
    \hline
    $\mathbf{D}$ & $\mathbf{D}$ & 0 & 0 \\
    \hline
    \end{tabular}
\]
Параметр $\mathbf{b}$ задаётся в начале игры и определяет её ход. Когда все пары агентов сыграют, каждый агент меняет свою стратегию на стратегию соседа с наибольшим счётом. Затем все шаги повторяются.

В нашем случае игра происходит на простой трёхмерной кубической решётке (у каждого агента 27 соседей, включая его самого). 

\section{Код}

\subsection{Базовая функция}

Базовая версия основной функции была написана на чистом питоне. Выполнение одного шага игры у этой версии функции на поле размером $60\times60\times60$ занимает примерно 16 секунд. Учитывая, что для замеров требовалось сделать более 10000 шагов на 1500 различных полях, это могло занять несколько месяцев. Для уменьшения времени работы функции я использовал Cython и Numba.

Все замеры времени проводились на суперкомпьютерном кластере Высшей Школы Экономики. В репозитории \ref{githab} находится код.

\subsection{Cython}

\href{https://cython.org/}{Cython} преобразует код написанный на Python в код си, и затем позволяет компилировать его и использовать коде на Python. Основным преимуществом данного метода является строгое определение типов переменных на уровне си, что значительно ускоряет выполнение операций с этими переменными. 


Код главной функции на Cython отличается от предыдущей версии только строго определёнными типами переменных. Время работы данной функции на поле $60\times60\times60$ 28 миллисекунд, что примерно в 600 раз быстрее версии на чистом питоне. 

\begin{figure}[H]
	 \centering
    	\includegraphics[width = \textwidth]{P_to_C.png}
    	\caption{Отношение времени работы функции на чистом Python и с Cython}
	\label{fig:Fcgraph}
\end{figure}

Стоит отметить, что при разработке в Jupyter notebook действительно было достаточно определить типы переменных и обернуть функцию в декоратор. Но для замеров на кластере было необходимо написать код запускаемы на питоне без каких-либо оболочек. В таком случае декораторы, предоставляемые Jupyter notebook использовать не получится, и функцию придётся компилировать на си с помощью отдельно написанного подготовительного скрипта и потом вызывать как библиотеку в основном коде. Помимо того, что подобные действия не требуются при использовании компилятора numba (который будет описан ниже), компиляция происходит под текущую платформу, из-за чего мне пришлось отдельно делать это на моём компьютере и на кластере, так как первый работает на Windows, а второй на Linux.



\subsection{Numba}

\href{http://numba.pydata.org/}{Numba} это компилятор на лету, основан на технологии LLVM. Он компилирует байт-кода питона в машинный код, за счёт чего и происходит ускорение.


Версия с Numba состоит из подготовительной функции, в которой заранее создаются нулевые массивы, и массив с игровым полем из трёхмерного переводится в одномерный, и основной функции, которая вызывается из подготовительной. Основная функция почти не отличается от функции на чистом питоне, но обёрнута в декоратор. Подготовительная функция нужна, чтобы не использовать numpy в обёртываемой функции, так как numba поддерживает не все операции из данной библиотеки. Время работы 36 миллисекунд

\begin{figure}[H]
	 \centering
    	\includegraphics[width = \textwidth]{P_to_N.png}
    	\caption{Отношение времени работы функции на чистом Python и с Numba}
	\label{fig:Fcgraph}
\end{figure}

\section{Итог}




Оба метода значительно ускоряют код. Несмотря на то, что Cython быстрее чем numba, отношение времени работы порядка единицы, при этом переделать функции под Numba значительно проще, особенно если в ней не используются другие библиотеки, нужно буквально написать одну строчку. Numba хорошо работает на разных платформах, в то время как функцию с Cython мне приходилось отдельно компилировать для Windows на своём компьютере и  для Linux на кластере. Использовать numba гораздо проще и удобнее, если не требуется предельно высокая скорость работы, данный метод является отличной альтернативой Cython. 

\begin{figure}[H]
	 \centering
    	\includegraphics[width = \textwidth]{C_to_N.png}
    	\caption{Отношение времени работы функции с Cython и с Numba}
	\label{fig:Fcgraph}
\end{figure}

\begin{figure}[H]
	 \centering
    	\includegraphics[width = \textwidth]{C_and_N.png}
    	\caption{ Время работы функций с Cython и с Numba}
	\label{fig:Fcgraph}
\end{figure}

\begin{thebibliography}{1}

\bibitem{githab}
\label{githab}
\href{https://github.com/MoskalenkoRomanBorisovich/Spatial-Distribution-Evolutionary-Game}{Репозиторий github}

\end{thebibliography}










\end{document}

